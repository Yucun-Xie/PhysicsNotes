\documentclass[12pt]{article}
%\usepackage[english]{babel}
\usepackage{graphicx}
%\usepackage{framed}
%\usepackage[normalem]{ulem}
\usepackage{indentfirst}
\usepackage{amsmath,amsthm,amssymb,amsfonts}
\numberwithin{equation}{subsection}
\usepackage[italicdiff]{physics}
\usepackage[T1]{fontenc}
\usepackage{wrapfig}
\usepackage{lmodern,mathrsfs}
\usepackage[inline,shortlabels]{enumitem}
\setlist{topsep=2pt,itemsep=2pt,parsep=0pt,partopsep=0pt}
\usepackage[dvipsnames]{xcolor}
\usepackage[utf8]{inputenc}
%\usepackage[letterpaper, top=0.5in,bottom=0.2in, left=0.5in, right=0.5in, footskip=0.3in, includefoot]{geometry}
\usepackage[letterpaper,left=0.5in, right=0.5in,top=0.7in,bottom=1in,footskip=0.3in,includefoot]{geometry}
\usepackage[most]{tcolorbox}
\usepackage{tikz,tikz-3dplot,tikz-cd,tkz-tab,tkz-euclide,pgf,pgfplots}
\pgfplotsset{compat=newest}
\usepackage{multicol}
\usepackage[bottom,multiple]{footmisc} 
%\usepackage[backend=bibtex,style=numeric]{biblatex}
%\addbibresource{bibliography}
\usepackage{hyperref}
\usepackage[nameinlink]{cleveref} 
\usepackage[titletoc, toc, page]{appendix}





\newtheoremstyle{mystyle}{}{}{}{}{\sffamily\bfseries}{.\\}{ }{}
\newtheoremstyle{cstyle}{}{}{}{}{\sffamily\bfseries}{}{ }{\thmnote{#3}}

\theoremstyle{mystyle}{\newtheorem{definition}{Definition}[subsection]}
%\theoremstyle{mystyle}{\newtheorem{proposition}[definition]{Proposition}}
\theoremstyle{mystyle}{\newtheorem{theorem}[definition]{Theorem}}
%\theoremstyle{mystyle}{\newtheorem{lemma}[definition]{Lemma}}
%\theoremstyle{mystyle}{\newtheorem{corollary}[definition]{Corollary}}
\theoremstyle{mystyle}{\newtheorem*{remark}{Remark}}
%\theoremstyle{mystyle}{\newtheorem*{remarks}{Remarks}}
\theoremstyle{mystyle}{\newtheorem{example}{Example}[subsection]}
\theoremstyle{mystyle}{\newtheorem{examples}{Examples}[subsection]}
%\theoremstyle{definition}{\newtheorem*{exercise}{Exercise}}
\theoremstyle{mystyle}{\newtheorem{cthm}{}[subsection]}


\tcolorboxenvironment{definition}{boxrule=0pt,boxsep=0pt,colback={red!10},left=8pt,right=8pt,enhanced jigsaw, borderline west={2pt}{0pt}{red},sharp corners,before skip=10pt,after skip=10pt,breakable}
%\tcolorboxenvironment{proposition}{boxrule=0pt,boxsep=0pt,colback={Orange!10},left=8pt,right=8pt,enhanced jigsaw, borderline west={2pt}{0pt}{Orange},sharp corners,before skip=10pt,after skip=10pt,breakable}
\tcolorboxenvironment{theorem}{boxrule=0pt,boxsep=0pt,colback={blue!10},left=8pt,right=8pt,enhanced jigsaw, borderline west={2pt}{0pt}{blue},sharp corners,before skip=10pt,after skip=10pt,breakable}
\tcolorboxenvironment{example}{boxrule=0pt,boxsep=0pt,colback={Green!10},left=8pt,right=8pt,enhanced jigsaw, borderline west={2pt}{0pt}{Green},sharp corners,before skip=10pt,after skip=10pt,breakable}
\tcolorboxenvironment{examples}{boxrule=0pt,boxsep=0pt,colback={violet!10},left=8pt,right=8pt,enhanced jigsaw, borderline west={2pt}{0pt}{violet},sharp corners,before skip=10pt,after skip=10pt,breakable}
\tcolorboxenvironment{proof}{boxrule=0pt,boxsep=0pt,blanker,borderline west={2pt}{0pt}{CadetBlue!80!white},left=8pt,right=8pt,sharp corners,before skip=10pt,after skip=10pt,breakable}
\tcolorboxenvironment{remark}{boxrule=0pt,boxsep=0pt,blanker,borderline west={2pt}{0pt}{Cyan},left=8pt,right=8pt,before skip=10pt,after skip=10pt,breakable}
%\tcolorboxenvironment{remarks}{boxrule=0pt,boxsep=0pt,blanker,borderline west={2pt}{0pt}{Green},left=8pt,right=8pt,before skip=10pt,after skip=10pt,breakable}
%\tcolorboxenvironment{example}{boxrule=0pt,boxsep=0pt,blanker,borderline west={2pt}{0pt}{Black},left=8pt,right=8pt,sharp corners,before skip=10pt,after skip=10pt,breakable}
%\tcolorboxenvironment{examples}{boxrule=0pt,boxsep=0pt,blanker,borderline west={2pt}{0pt}{Black},left=8pt,right=8pt,sharp corners,before skip=10pt,after skip=10pt,breakable}
\tcolorboxenvironment{cthm}{boxrule=0pt,boxsep=0pt,colback={orange!10},left=8pt,right=8pt,enhanced jigsaw, borderline west={2pt}{0pt}{orange},sharp corners,before skip=10pt,after skip=10pt,breakable}
\tcolorboxenvironment{equation}{boxrule=0pt,boxsep=0pt,colback={Magenta!10},left=8pt,right=8pt,enhanced jigsaw, borderline west={2pt}{0pt}{Magenta},sharp corners,before skip=10pt,after skip=10pt,breakable}
\tcolorboxenvironment{align}{boxrule=0pt,boxsep=0pt,colback={Cyan!10},left=8pt,right=8pt,enhanced jigsaw, borderline west={2pt}{0pt}{Cyan},sharp corners,before skip=10pt,after skip=10pt,breakable}



\usepackage[explicit]{titlesec}
\titleformat{\section}{\fontsize{24}{30}\sffamily\bfseries}{\thesection}{20pt}{#1}
\titleformat{\subsection}{\fontsize{16}{18}\sffamily\bfseries}{\thesubsection}{12pt}{#1}
\titleformat{\subsubsection}{\fontsize{10}{12}\sffamily\large\bfseries}{\thesubsubsection}{8pt}{#1}

\titlespacing*{\section}{0pt}{5pt}{5pt}
\titlespacing*{\subsection}{0pt}{5pt}{5pt}
\titlespacing*{\subsubsection}{0pt}{5pt}{5pt}

%\newcommand{\sectionbreak}{\clearpage} %Start every section on a new page
\newcommand{\tbf}[1]{\textbf{#1}}
\newcommand{\p}{\partial}
\newcommand{\id}{\mathrm{d}}
%\newcommand{\Disp}{\displaystyle}
%\newcommand{\qe}{\hfill\(\bigtriangledown\)}
%\DeclareMathAlphabet\mathbfcal{OMS}{cmsy}{b}{n}
%\setlength{\parindent}{0.2in}
%\setlength{\parskip}{0pt}
%\setlength{\columnseprule}{0pt}

\title{\huge\sffamily\bfseries Quantum Field Theory}
\author{\Large\sffamily Yucun Xie}
\date{\sffamily \today}

\begin{document}

\setlength{\abovedisplayskip}{3pt}
\setlength{\belowdisplayskip}{3pt}
\setlength{\abovedisplayshortskip}{0pt}
\setlength{\belowdisplayshortskip}{0pt}
\maketitle

%Custom colors for different environments
\definecolor{contcol1}{HTML}{72E094}
\definecolor{contcol2}{HTML}{24E2D6}
\definecolor{convcol1}{HTML}{C0392B}
\definecolor{convcol2}{HTML}{8E44AD}

\begin{tcolorbox}[
    title=Contents, fonttitle=\huge\sffamily\bfseries\selectfont,
    interior style={left color=contcol1!40!white,right color=contcol2!40!white},
    frame style={left color=contcol1!80!white,right color=contcol2!80!white},
    coltitle=black,top=2mm,bottom=2mm,left=2mm,right=2mm,drop fuzzy shadow,enhanced,breakable]
  \makeatletter
  \@starttoc{toc}
  \makeatother
\end{tcolorbox}

\newpage
\begin{tcolorbox}[
    title=Conventions, fonttitle=\large\sffamily\bfseries\selectfont,
    interior style={left color=convcol1!40!white,right color=convcol2!40!white},
    frame style={left color=convcol1!80!white,right color=convcol2!80!white},
    coltitle=black,top=2mm,bottom=2mm,left=2mm,right=2mm,drop fuzzy shadow,enhanced,breakable]
  \begin{enumerate}

    \item Greek index (e.g. $\alpha, \beta, \mu, \nu$) take value from \{0, 1, 2, 3\}.
          %\item Events denoted by cursive capitals  (e.g. $\mathscr{A}, \mathscr{B}, \mathscr{E}$).
    \item $(x^0, x^1, x^2, x^3) \equiv (t, x, y, z) \equiv x^{\alpha}$.
    \item Latin index (e.g.$ i, j, k$) take value from \{1, 2, 3\}.
    \item Natural units ($c=1$).
    \item Einstein summation convention. \[ds^2 = g_{\mu \nu} dx^{\mu} dx^{\nu}=
            \sum_{\mu=0}^{3} \sum_{\nu=0}^{3}g_{\mu \nu} dx^{\mu} dx^{\nu}\]
    \item Metric signature $(+, -, -, -)$.

  \end{enumerate}
\end{tcolorbox}

\newpage

%begin here ---------------------------------------------------------------------------------------------
\section{Scalar Field Theory}

\subsection{Canonical Quantization}

\section{Quantum Electrodynamics}

%\section{Renormalization}

\newpage
\appendix
\addcontentsline{toc}{section}{Appendix~}
\section{Classical Field Theory}
Classical field theory is something you are supposed to know for QFT but is never formally taught, so I give a brief overview here.
Field theory is very similar to Lagrangian mechanics; instead of the usual system,
we now have a spacetime-dependent \textbf{fields} \(\Phi^i(x^\mu)\) (the \(i\) here is not tensor index),
and the action becomes a \textbf{functional} of these fields. A functional is a ``function of function'', which map a function to a number,
note that the functional is not simply a composition function, which maps number to number. \par
In field theory, the Lagrangian is usually expressed as an integral over \textbf{Lagrange density},
which are a function of the fields and their derivative.
\begin{cthm}
  \[L=\int d^3x\;\mathcal{L}(\Phi^i,\p_{\mu}\Phi^i)\]
\end{cthm}
Then the action
\begin{cthm}
  \[S=\int dt\;L = \int d^4x\;\mathcal{L}(\Phi^i,\p_{\mu}\Phi^i)\]
\end{cthm}
By varying the action, just like what we did in classical mechanics, we can obtain the equation of motion of field:
\begin{cthm}
  \begin{align*}
    \delta S & = \int d^4x\left[\frac{\p\mathcal{L}}{\p \Phi^i}\delta\Phi^i+
    \frac{\p\mathcal{L}}{\p(\p_\mu \Phi^i)}\p_\mu (\delta\Phi^i)\right]      \\
             & =\int d^4x\left[\frac{\p\mathcal{L}}{\p \Phi^i}-\p_\mu
      \frac{\p\mathcal{L}}{\p(\p_\mu \Phi^i)}\right]\delta\Phi^i
  \end{align*}
\end{cthm}
We have obtained the field version of \textbf{Euler-Lagrange equation}
\begin{cthm}
  \[\frac{\p\mathcal{L}}{\p \Phi^i}-\p_\mu
    \frac{\p\mathcal{L}}{\p(\p_\mu \Phi^i)}=0\]
\end{cthm}
Let's consider a simple example, the real scalar field \[\phi:\ x^{\mu} \rightarrow \mathbb{R}\].
The contribution of action are

\begin{enumerate}
  \item Kinetic term: \(\frac{1}{2}\dot{\phi}^2\)
  \item Gradient term: \(\frac{1}{2}(\nabla\phi)^2\)
  \item potential term: \(V(\phi)\)
\end{enumerate}
We could combine them into a Lorentz-invariant Lagrange density:
\begin{cthm}
  \[\mathcal{L}= \frac{1}{2}\eta^{\mu\nu}(\p_{\mu}\phi)(\p_{\nu}\phi)-V(\phi)\]
\end{cthm}
Apply the Euler-Lagrange equation; we then get the equation of motion:
\begin{cthm}
  \begin{align*}            & \eta^{\mu\nu}\p_{\mu}\p_{\nu}\phi+\frac{\mathrm{d}V(\phi)}{\mathrm{d}\phi} \\
                      =\  & \square\phi+\frac{\mathrm{d}V(\phi)}{\mathrm{d}\phi}                       \\
                      =\  & 0
  \end{align*}
  Where \(\square\) is \textbf{d'Alembertian}.
\end{cthm}
\subsection{Klein-Gordon Field}
If the scalar field is massive, the potential \(V(\phi)=\frac{1}{2}m^2\phi^2\),
we obtained the \textbf{Klein-Gordon equation}:
\begin{cthm}
  \[(\square +m^2)\phi = 0\]
\end{cthm}
You will see this equation again and again in quantum field theory.
\begin{remark}
  The Klein-Gordon field could be analog to infinite many coupled infinitesimal harmonic oscillators;
  each ``mass'' is affected by neighboring springs and has its kinetic energy.
\end{remark}



\subsection{Hamiltonian Field Theory}
The Lagrangian field theory is naturally Lorentz invariant. However, we also need Hamiltonian formalism for the field theory
because it is easier to transition from quantum mechanics.
Let's first consider the classical definition of Hamiltonian, \[H=\sum p\dot{q}-L\].
From the definition of Lagrangian, we can derive the conjugate momentum as follow:
\begin{cthm}
  \begin{align*}
    p(x^i)\equiv & \;\frac{\p L}{\p \dot{\phi}(x^i)}
    =\frac{\p}{\p \dot{\phi}(x^i)}\int \mathrm{d}^3x' \mathcal{L} \\
    \propto      & \;\pi(x^i) \mathrm{d}^3x
  \end{align*}
  \bigskip
  Where \[\pi \equiv \frac{\p \mathcal{L}}{\p \dot{\phi}}\]
  is called \textbf{momentum density}
\end{cthm}
Therefore, the Hamiltonian for field theory is:
\begin{definition}
  \begin{align*}
    H=&\sum p(x^i)\phi(x^i)-L\\
    =& \int \id^3x\left[\pi \dot{\phi}-\mathcal{L}\right]\\
    =& \int \id^3x\;\mathcal{H}
  \end{align*}
  Where \(\mathcal{H}\) is called \tbf{Hamiltonian density}.
\end{definition}

\begin{example}
  As a simple example, the Hamiltonian for K-G field is:
  \[H=\int\id^3x\;\mathcal{H}=\int\id^3x\left[\frac{1}{2}\pi^2+\frac{1}{2}(\nabla\phi)^2+\frac{1}{2}m^2\phi^2\right]\]
\end{example}

\subsection{Noether's Theorem}
A Lagrangian may be invariant under some special type of transformation.
For example, a Lagrangian for a complex scalar field
\[\mathcal{L}= |\p_\mu \phi|^2 -m^2|\phi|^2\]
is invariant under transformation \(\phi \rightarrow e^{i\alpha}\phi\).
We call this transformation a \textbf{symmetry} of the Lagrangian.
When the parameter of transformation (\(\alpha\) for this case) can be taken infinitesimal,
we say this symmetry is \textbf{continuous}.
We can explicit see that \(\frac{\delta \mathcal{L}}{\delta \alpha}=0\).
\par By applying the Euler-Lagrange equation, we can deduce that \(\p_\mu J_{\mu}=0\),
where \(J_{\mu}\) is defined as follow:
\begin{definition}
  \[J_{\mu}= \sum_n\frac{\p \mathcal{L}}{\p(\p_\mu \phi_n)}\frac{\delta\phi_n}{\delta \alpha}\]\\
  \(J_{\mu}\) is known as \textbf{Noether current} or \textbf{conserved current}
\end{definition}

\begin{example}
  For the above complex scalar field, we can calculate the conserved current:
  \bigskip
  \begin{align*}
    J_{\mu}= & \sum_n\frac{\p \mathcal{L}}{\p(\p_\mu \phi_n)}\frac{\delta\phi_n}{\delta \alpha} \\
    =        & \frac{\p \mathcal{L}}{\p(\p_\mu \phi)}\frac{\delta\phi}{\delta \alpha}
    +\frac{\p \mathcal{L}}{\p(\p_\mu \phi^*)}\frac{\delta\phi^*}{\delta \alpha}                 \\
    =        & -i\left[\phi\p_\mu\phi^*-\phi^* \p_\mu\phi\right]
  \end{align*}
  We can check that \(\p_\mu J_{\mu}=0\) explicitly:
  \begin{align*}
    \p_\mu J_{\mu}= & -i\left[\phi\square\phi^*-\phi^* \square\phi\right] \\
    =               & 0
  \end{align*}
  Where we applied the equation of motion in the last step.
\end{example}
We call \(J_\mu\) conserved current because we can find conserved quantity by integral over its 0-component:
\begin{definition}
  \[Q=\int d^3x\;J_0\]
  Where the \(Q\) is called \textbf{conserved charge} or \textbf{Noether's charge}
\end{definition}[Conserved charge]
We say \(Q\) is conserved because
\begin{cthm}
  \[\p_tQ=\int d^3x \; \p_t J_0 = \int d^3x\;\vec{\nabla}\cdot \vec{J} = 0.\]
\end{cthm}
The above argument is called \textbf{Noether's theorem}.
\begin{theorem}[Noether's theorem]
  If a Lagrangian has a continuous symmetry,
  then exists a current associated with that symmetry that is
  conserved when the equations of motion are satisfied.
\end{theorem}

\subsection{Energy-momentum Tensor}
The physics at spacetime point \(x\) should have same form at spacetime point \(y\),
this argument arises a symmetry called \textbf{specetime translational symmetry}.
By Noether's theorem, we can find the four conserved currents for this symmetry for an infinitesimal
spacetime translation \(\xi^\mu\):
\begin{cthm}
  \[T_{\mu\nu}=\sum_n \frac{\p \mathcal{L}}{\p(\p_{\mu}\phi_n)}\p_\nu\phi_n-g_{\mu\nu}\mathcal{L}\]

  \(T_{\mu\nu}\) is called \textbf{energy-momentum tensor}.
\end{cthm}
Four conserved currents correspond to values of \(\nu\), and four conserved charges are energy and momentum.
We can see that the energy density is the 00-component of \(T_{\mu\nu}\).



\newpage





\end{document}