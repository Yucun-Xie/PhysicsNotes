\documentclass[12pt]{article}
%\usepackage[english]{babel}
\usepackage{graphicx}
%\usepackage{framed}
%\usepackage[normalem]{ulem}
\usepackage{indentfirst}
\usepackage{amsmath,amsthm,amssymb,amsfonts}
\numberwithin{equation}{section}
\usepackage{mathtools}
\usepackage[italicdiff]{physics}
\usepackage[T1]{fontenc}
\usepackage{lmodern,mathrsfs}%font
\usepackage[inline,shortlabels]{enumitem}
\setlist{topsep=2pt,itemsep=2pt,parsep=0pt,partopsep=0pt}
\usepackage[dvipsnames]{xcolor}
\usepackage[utf8]{inputenc}
%\usepackage[letterpaper, top=0.5in,bottom=0.2in, left=0.5in, right=0.5in, footskip=0.3in, includefoot]{geometry}
\usepackage[letterpaper,left=0.5in, right=0.5in,top=0.7in,bottom=1in,footskip=0.3in,includefoot]{geometry}
\usepackage[most]{tcolorbox}
\usepackage{tikz,tikz-3dplot,tikz-cd,tkz-tab,tkz-euclide,pgf,pgfplots}
\pgfplotsset{compat=newest}
\usepackage{multicol}
\usepackage[bottom,multiple]{footmisc} 
%\usepackage[backend=bibtex,style=numeric]{biblatex}
%\addbibresource{bibliography}
\usepackage{hyperref}
\hypersetup{
    colorlinks=true,
    linkcolor=magenta,
    filecolor=pink,      
    urlcolor=cyan,
    pdftitle={QFTCS Note},
    pdfpagemode=FullScreen,
    }
\usepackage[titletoc, toc, page]{appendix}
\usepackage{subfiles}



\newtheoremstyle{1style}{}{}{}{}{\sffamily\bfseries}{.\\}{ }{}
\newtheoremstyle{2style}{}{}{}{}{\sffamily\bfseries}{\\}{ }{}
%\newtheoremstyle{cstyle}{}{}{}{}{\sffamily\bfseries}{}{ }{\thmnote{#3}}

\theoremstyle{1style}
\newtheorem{definition}[equation]{Definition}
\newtheorem{theorem}[equation]{Theorem}
\newtheorem{example}[equation]{Example}
\newtheorem{cthm}[equation]{}
\newtheorem*{remark}{Remark}


%\theoremstyle{mystyle}{\newtheorem{proposition}[definition]{Proposition}}
%\theoremstyle{mystyle}{\newtheorem{lemma}[definition]{Lemma}}
%\theoremstyle{mystyle}{\newtheorem{corollary}[definition]{Corollary}}
%\theoremstyle{mystyle}{\newtheorem*{remarks}{Remarks}}
%\newtheorem{examples}{Examples}[equation]
%\theoremstyle{definition}{\newtheorem*{exercise}{Exercise}}



\tcolorboxenvironment{definition}
{boxrule=0pt,boxsep=0pt,colback={red!10},left=8pt,right=8pt,enhanced jigsaw, borderline west={2pt}{0pt}{red},
sharp corners,before skip=10pt,after skip=10pt,breakable}
\tcolorboxenvironment{theorem}
{boxrule=0pt,boxsep=0pt,colback={MidnightBlue!10},left=8pt,right=8pt,enhanced jigsaw, borderline west={2pt}{0pt}{MidnightBlue},
sharp corners,before skip=10pt,after skip=10pt,breakable}
\tcolorboxenvironment{example}
{boxrule=0pt,boxsep=0pt,colback={Magenta!10},left=8pt,right=8pt,enhanced jigsaw, borderline west={2pt}{0pt}{Magenta},
sharp corners,before skip=10pt,after skip=10pt,breakable}
\tcolorboxenvironment{proof}
{boxrule=0pt,boxsep=0pt,blanker,borderline west={2pt}{0pt}{CadetBlue!80!white},left=8pt,right=8pt,
sharp corners,before skip=10pt,after skip=10pt,breakable}
\tcolorboxenvironment{cthm}
{boxrule=0pt,boxsep=0pt,colback={orange!10},left=8pt,right=8pt,enhanced jigsaw, borderline west={2pt}{0pt}{orange},
sharp corners,before skip=10pt,after skip=10pt,breakable}
\tcolorboxenvironment{equation}
{boxrule=0pt,boxsep=0pt,colback={Green!10},left=8pt,right=8pt,enhanced jigsaw, borderline west={2pt}{0pt}{Green},
sharp corners,before skip=10pt,after skip=10pt,breakable}
\tcolorboxenvironment{align}
{boxrule=0pt,boxsep=0pt,colback={blue!10},left=8pt,right=8pt,enhanced jigsaw, borderline west={2pt}{0pt}{blue},
sharp corners,before skip=10pt,after skip=10pt,breakable}
\tcolorboxenvironment{remark}
{boxrule=0pt,boxsep=0pt,blanker,borderline west={2pt}{0pt}{Cyan},left=8pt,right=8pt,before skip=10pt,after skip=10pt,breakable}

%\tcolorboxenvironment{proposition}{boxrule=0pt,boxsep=0pt,colback={Orange!10},left=8pt,right=8pt,enhanced jigsaw, borderline west={2pt}{0pt}{Orange},sharp corners,before skip=10pt,after skip=10pt,breakable}
%\tcolorboxenvironment{examples}{boxrule=0pt,boxsep=0pt,colback={violet!10},left=8pt,right=8pt,enhanced jigsaw, borderline west={2pt}{0pt}{violet},sharp corners,before skip=10pt,after skip=10pt,breakable}
%\tcolorboxenvironment{remarks}{boxrule=0pt,boxsep=0pt,blanker,borderline west={2pt}{0pt}{Green},left=8pt,right=8pt,before skip=10pt,after skip=10pt,breakable}
%\tcolorboxenvironment{example}{boxrule=0pt,boxsep=0pt,blanker,borderline west={2pt}{0pt}{Black},left=8pt,right=8pt,sharp corners,before skip=10pt,after skip=10pt,breakable}
%\tcolorboxenvironment{examples}{boxrule=0pt,boxsep=0pt,blanker,borderline west={2pt}{0pt}{Black},left=8pt,right=8pt,sharp corners,before skip=10pt,after skip=10pt,breakable}





\usepackage[explicit]{titlesec}
\titleformat{\section}{\fontsize{24}{30}\sffamily\bfseries}{\thesection}{20pt}{#1}
\titleformat{\subsection}{\fontsize{16}{18}\sffamily\bfseries}{\thesubsection}{12pt}{#1}
\titleformat{\subsubsection}{\fontsize{10}{12}\sffamily\large\bfseries}{\thesubsubsection}{8pt}{#1}

\titlespacing*{\section}{0pt}{5pt}{5pt}
\titlespacing*{\subsection}{0pt}{5pt}{5pt}
\titlespacing*{\subsubsection}{0pt}{5pt}{5pt}

\newcommand{\sectionbreak}{\clearpage} %Start every section on a new page
\newcommand{\tbf}[1]{\textbf{#1}}
\newcommand{\p}{\partial}
\newcommand{\id}{\mathrm{d}}
%\newcommand{\Disp}{\displaystyle}
%\newcommand{\qe}{\hfill\(\bigtriangledown\)}
%\DeclareMathAlphabet\mathbfcal{OMS}{cmsy}{b}{n}
%\setlength{\parindent}{0.2in}
%\setlength{\parskip}{0pt}
%\setlength{\columnseprule}{0pt}

\title{\huge\sffamily\bfseries Quantum Field Theory}
\author{\Large\sffamily Yucun Xie}
\date{\sffamily \today}

\begin{document}

\setlength{\abovedisplayskip}{3pt}
\setlength{\belowdisplayskip}{3pt}
\setlength{\abovedisplayshortskip}{0pt}
\setlength{\belowdisplayshortskip}{0pt}
\maketitle

%Custom colors for different environments
\definecolor{contcol1}{HTML}{72E094}
\definecolor{contcol2}{HTML}{24E2D6}
\definecolor{convcol1}{HTML}{C0392B}
\definecolor{convcol2}{HTML}{8E44AD}

\begin{tcolorbox}[
    title=Contents, fonttitle=\huge\sffamily\bfseries\selectfont,
    interior style={left color=contcol1!40!white,right color=contcol2!40!white},
    frame style={left color=contcol1!80!white,right color=contcol2!80!white},
    coltitle=black,top=2mm,bottom=2mm,left=2mm,right=2mm,drop fuzzy shadow,enhanced,breakable]
  \makeatletter
  \@starttoc{toc}
  \makeatother
\end{tcolorbox}

\newpage










\begin{tcolorbox}[
    title=Conventions, fonttitle=\large\sffamily\bfseries\selectfont,
    interior style={left color=convcol1!40!white,right color=convcol2!40!white},
    frame style={left color=convcol1!80!white,right color=convcol2!80!white},
    coltitle=black,top=2mm,bottom=2mm,left=2mm,right=2mm,drop fuzzy shadow,enhanced,breakable]
  \begin{enumerate}

    \item Greek index (e.g. $\alpha, \beta, \mu, \nu$) run over time and space.
          %\item Events denoted by cursive capitals  (e.g. $\mathscr{A}, \mathscr{B}, \mathscr{E}$).
    \item Latin index (e.g. $ i, j, k$) run over space.
    \item Natural units ($c=\hbar=1$).
    \item Einstein summation convention. \[ds^2 = g_{\mu \nu} dx^{\mu} dx^{\nu}=
            \sum_{\mu=0}^{n-1} \sum_{\nu=0}^{n-1}g_{\mu \nu} dx^{\mu} dx^{\nu}\]
    \item Metric signature $(+, -, -, -)$.

  \end{enumerate}
\end{tcolorbox}

\newpage





%begin here ---------------------------------------------------------------------------------------------
\section{Field Theory}
\subsection{Necessity of Field}
Quantum field theory is the combination of quantum mechanics and relativistic field theory, 
instead of studying the dynamics of particles, we study the dynamics of the field.
To see why we can't simply stick with relativistic quantum mechanics, consider the relativistic 
wave equation, for example, \tbf{Klein-Gordon equation}:
\begin{equation}
  (\Box +m^2)\phi=0
\end{equation}



\subsection{Canonical Quantization}

\section{Quantum Electrodynamics}

%\section{Renormalization}

\newpage
\appendix
\addcontentsline{toc}{section}{Appendix~}

\section{Classical Field Theory}
\subfile{Appendix/Classical_Field_Theory}


\newpage





\end{document}